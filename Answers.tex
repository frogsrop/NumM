\documentclass[16pt]{article}
\usepackage[utf8]{inputenc}
\usepackage[russian]{babel}
\usepackage{amsmath}
\usepackage{amssymb}
\usepackage{upgreek}
\usepackage{graphicx}
\usepackage[dvipsnames]{xcolor}
\usepackage{tikz}
\usepackage[export]{adjustbox}                  
\usepackage[hidelinks]{hyperref}

\title {Конспектик по численным методам}
\author {Шоколадные бойцы}

\begin {document}
	\pagenumbering {gobble}
	\maketitle
  
	\newpage 
		\pagenumbering {arabic}
		\section {Понятие погрешности. Абсолютная и относительная погрешности.}	
			Погрешность - отклонение величины от ее истинного значения. То есть, если $a$ реальное значение, а $a^{*}$ приближенное значение, то погрешность будет равна $a-a^{*}$
			\subsection {Причины:}
				\begin {enumerate}
				\item {
					Неустранимая погрешность - 
					$\begin {cases}
						\text{Математическая модель является лишь}\\
						\text{\qquad приближенным описание процеса.}\\
						\text{Исходные данные содержат погрешности.}\\
					\end {cases}$
				}
				\item {Погрершность метода - применяемые методы зачастую являются приближенными.}
				\item {Вычислительная погрешность - при действиях с числами происходит округление.}
				\end {enumerate}
			\subsection {Абсолютная и относительная погрешности}	
				Абсолютная погрешность: $\Delta(a^{*}) = |a-a^{*}|$\\
				Относительная погрешность: $\delta(a^{*}) = \frac{|a-a^{*}|}{|a|} = \frac{\Delta(a^{*})}{|a|}$\\
				Чаще реальное значение неизвестно поэтому используются оценки погрешности.\\
				$\Delta(a^{*}) \leq \overline{\raisebox{3mm}{} \Delta}(a^{*})$,\\
				$\delta(a^{*}) \leq \overline{\raisebox{3mm}{} \delta}(a^{*})$,\\
				где $\overline{\raisebox{3mm}{} \Delta}(a^{*})$ и $\overline{\raisebox{3mm}{} \delta}(a^{*})$ - верхняя граница абсолютной и отностельной погрешности соответственно. При этом если одна из величин известна, вторую можно выразить через нее.
				На практике часто используются следующие приближения:\\
				$\overline{\raisebox{3mm}{}\delta}(a^{*}) \approx \frac{\overline{\raisebox{3mm}{} \Delta}(a^{*})}{|a^{*}|}$,\\				 
				$\overline{\raisebox{3mm}{} \Delta}(a^{*}) \approx |a^{*}| \overline{\raisebox{3mm}{} \delta}(a^{*})$
		\section {Погрешности арифметических операций и функций.}
			\subsection {Сумма и разность}
				Для абсолютной погрешности:
				Абсолютная погрешность суммы и разности не превосходит суммы абсолютных погрешностей.
				$\Delta(a^{*} \pm b^{*}) \leq \Delta(a^{*}) + \Delta(b^{*})$\\
				Оценка суммы и разности
				$\overline{\raisebox{3mm}{} \Delta}(a^{*} \pm b^{*}) = 
				\overline{\raisebox{3mm}{} \Delta}(a^{*}) + \overline{\raisebox{3mm}{} \Delta}(b^{*})$\\
				Для относительной погрешности:
				Относительная погрешность суммы не превосходит максимальной относительной погрешности.\\
				$\delta (a^{*} + b^{*}) \leq \max(\delta(a^{*}),\delta(b^{*}))$\\
				Относительная погрешность разности выглядит так: \\
				$\delta (a^{*} - b^{*}) \leq \max(\delta(a^{*}),\delta(b^{*})) * \frac{|a+b|}{|a-b|}$\\
				Для оценок относительной погрешности имеем аналогичные неравенства.
			\subsection {Произведение и частное.}
				Для абсолютной погрешности:\\
				Абсолютная погрешность произведения и частного не превышает суммы абсолютных погрешностей.\\
				$\Delta(a^{*} * b^{*}) \leq \Delta(a^{*}) + \Delta(b^{*})$\\
				$\Delta(a^{*} / b^{*}) \leq \Delta(a^{*}) + \Delta(b^{*})$\\
				Оценка произведения и частного:\\
				$\delta(a^{*}*b^{*}) \approx (\overline{\raisebox{3mm}{} \delta}(a^{*}) + \overline{\raisebox{3mm}{} \delta}(b^{*}))*|a^{*}*b^{*}|$\\
				$\delta(a^{*}/b^{*}) \approx (\overline{\raisebox{3mm}{} \delta}(a^{*}) + \overline{\raisebox{3mm}{} \delta}(b^{*}))*|a^{*}/b^{*}|$\\
				Для относительной погрешности:\\
				$\delta (a^{*}b^{*}) \leq \delta(a^{*}) + \delta(b^{*}) + \delta(a^{*})\delta(b^{*})$\\
				$\delta (a^{*}/b^{*}) \leq \frac{\delta(a^{*}) + \delta(b^{*})}{1 - \delta(b^{*})}$\\
				Оценка произведения и частного:\\
				$\overline{\raisebox{3mm}{} \delta}(a^{*}*b^{*}) \approx \overline{\raisebox{3mm}{} \delta}(\frac{a^{*}}{b^{*}}) \approx \overline{\raisebox{3mm}{} \delta}(a^{*}) + \overline{\raisebox{3mm}{} \delta}(b^{*})$
			\subsection {Функции.}
				Пусть $f(x) = f(x_1,\ x_2,\ x_3,\ ...,\ x_m)$ функция от $m$ переменных, дифференцируемая в области G, вычисление которой производится на $x^{*}_1,\ x^{*}_2,\ x^{*}_3,\ ...,\ x^{*}_m$.\\
				\subsubsection{Абсолютная погрешность.}
				$\Delta(y^{*}) \leq \sum_{j=1}^{m} \underset{[x,\ x^{*}]}{max}|f'_{x_j}|\Delta(x^{*}_j)$, где $[x,\ x^{*}]$ отрезок из $x$ d $x^{*}$\\
				Для оценки верно следующее:\\
				$\overline{\raisebox{3mm}{} \Delta}(y^{*})\approx \sum_{j=1}^{m} |f'_{x_j}(x^{*})|\overline{\raisebox{3mm}{} \Delta}(x^{*}_j)$\\
				$\overline{\raisebox{3mm}{} \Delta}(y^{*})\approx \sum_{j=1}^{m} |f'_{x_j}(x)|\overline{\raisebox{3mm}{} \Delta}(x^{*}_j)$\\
				\subsubsection{Относительная погрешность.}
				$\overline{\raisebox{3mm}{} \delta}(y^{*}) \approx \sum_{j=1}^{m} v^{*}_j \overline{\raisebox{3mm}{} \delta}(x^{*}_j)$\\
				$\overline{\raisebox{3mm}{} \delta}(y^{*}) \approx \sum_{j=1}^{m} v_j \overline{\raisebox{3mm}{} \delta}(x^{*}_j)$\\
				где\\
				$v^{*}_j = \frac{|x^{*}_j||f'_{x_j}(x^{*})|}{|f(x^{*})|}$\\
				$v^{*}_j = \frac{|x_j||f'_{x_j}(x)|}{|f(x)|}$\\
				\subsubsection{Неявные функции.}
				Пусть $F(y,\ x_1,\ x_2,\ ...,\ x_m) = 0 $ неявно заданная функция. Тогда $f'_{x_j}(x) = \frac{-F'_{x_j}}{F'_y}|_{y=f(x)},\ j=1,\ 2,\ ...,\ m.$\\
				Далее воспользуемся погрешностями описанными выше.\\ 
		\section {Связь погрешности и количества верных значащих цифр в позиционной записи вещественных чисел.}
			Если число $a^{*}$ содержит $N$ верных значащих цифр, то справедливо:\\
			$\delta(a^{*}) \leq (10^{N-1} - 1)^{-1} \approx 10^{-N+1}$\\
			Для того чтобы число $a^{*}$ содержало N верных значащих цифр, требуется:\\
			$\delta(a^{*}) \leq (10^{N} + 1)^{-1} \approx 10^{-N}$\\
			Если $a^{*}$ имеет ровно $N$ верных значащих цифр, то $10^{-N-1} \lesssim \delta(a^{*}) \lesssim 10^{-N+1}$, а так же $\delta(a^{*}) \sim 10^{-N}$ 
		\section {Компьютерное представление чисел, погрешность компьютерного округления.}
			Целые числа представляются как:\\
			$n = \pm(a_L2^L+...+a_12^1+a_02^0)$, где $a\in \{0,\ 1\}$\\
			или через дополнение.\\
			Вещественные числа представляются как:\\
			$n = \pm(a_12^{-1} + a_22^{-2} + ... + a_L2^{-L})*2^p$, где $a\in \{0,\ 1\}$\\
			$(a_12^{-1} + a_22^{-2} + ... + a_L2^{-L})$ эта часть называется мантисой.\\
			$p$ называют порядком.\\
			Число $n$ нормализуется так, чтобы выполнялось $a_1=1$.\\
			\subsection{Свойства и замечания:}
			\begin{enumerate}
			\item{На компьютере представим конечный набор рациональных чисел специального вида.}	
			\item{Диапазон мантисы и порядка ограничены. $0.5 \leq |m| < 1$, $|p|\leq 2^{L+1} - 1$}
			\item{Нельзя представить слишком большие и слишком маленькие числа в виду ограничения на порядок.}
			\item{Арифметические операции над числами портят точность. Абсолютную погрешность можно оценить как $|f(a,\ b)|*\upvarepsilon_M$}, где $f$ - арифметическая операция, а $\upvarepsilon_M$ относительная точность ЭВМ.
			\item{Можно в два раза увеличить размер мантисы.}
			\item{Удобно принимать так $1 + \upvarepsilon_M > 1$}
			\end{enumerate}
		\section{Понятия корректности, устойчивости и обусловленности вычислительных задач. Примеры хорошо и плохо обусловленных задач.}
			Вычислительная задача - одно из трех:
			\begin{enumerate}
				\item{Прямая задача}
				\item{Обратная задача}
				\item{Задача идентификации}
			\end{enumerate}
			Постановка задачи:
			\begin{enumerate}
				\item{Задание множества допустимых X (входных данных)}
				\item{Задание множества допустимых Y (выходных данных)}
			\end{enumerate}
			\subsection{Корректность задачи.}
			Задача корректна если:
			\begin{enumerate}
				\item{$\forall x\in X \exists y\in Y$}
				\item{Решение единственно}
				\item{Решнеие устойчиво по отношению к малым возмущениям входных данных}
			\end{enumerate}
			\subsection{Устойчивость решения.} 
			$\forall \upvarepsilon \exists \delta=\delta(\upvarepsilon)>0$ такое что $\forall x^{*}$ удовлетворяющих условию $\Delta(x^{*})<\delta$ выполняется $\exists y^{*}: \Delta(y^{*})<\upvarepsilon$\\
			Относительная устойчивость.\\
			Все $\Delta$ следует заменить на $\delta$\\
			Замечание!\\
			Некоторые задачи дифференцирования или суммирования ряда, не являются корректными, тем не менее имею практическую важность.\\
			\subsection{Обусловленность вычислительной задачи.}
			Под обусловленностью задачи понимается чувствительность решения к малым погрешностям входных данны.\\
			Так например, если при малых погрешностях входных данных, решение дает малые погрешнсоти, то говорят, что задача хорошо обусловленна. И наоборот, если при малых погрешностях могут быть сильные изменения решения.\\
			Число обусловленности - мера степени обусловленности задачи.\\
			Пусть выполняется неравенство:\\
			$\Delta (y^{*}) \leq v_\Delta \Delta(x^{*})$\\
			Тогда величина $v_\Delta$ называется абсолютным чистом обусловленности.\\
			Пусть выполняется неравенство:\\
			$\delta (y^{*}) \leq v_\delta \delta(x^{*})$\\
			Тогда величина $v_\delta$ называется относительным числом обусловленности.\\
			\subsection{ПРИМЕРЫ:}
			\subsubsection{Плохо обусловеленные}
			Пусть требуется найти корни многочлена:\\
			$P(x) = (x-1)(x-2)...(x-20)=x^{20} - 210 x^{19} + ... $. Эта задача устойчива. Но если менять первый коэффициент относительное число обусловленности очень большое.\\ \\
			$(x-1)=0$ ошибка в младшем коэффициенте приведет к комплексным корням, что говорит о плохой обусловленности.
			\subsubsection{Хорошо обусловеленные}
			$y = e^x$\\ \\
			$\int_{a}^{b}f(x) dx$ если у f постоянный знак на $[a,\ b]$, то задача хорошо обусловлена. 
\end {document}
